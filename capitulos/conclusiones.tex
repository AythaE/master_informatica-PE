\chapter{Conclusiones}

Como conclusiones de esta memoria me gustaría apuntar algunas cosas sobre el proceso de prácticas, en primer lugar me parece bastante poco flexible la selección de empresas para realizar las prácticas, ya que en la reunión informativa se comentó que solo se podían seleccionar empresas del listado oficial o bien unas prácticas ÍCARO. Esto imposibilita el contacto de los estudiantes con otras posibles empresas que les puedan interesar, lo que también favorece el acercar empresas a la universidad, ya que si un estudiante acude a una empresa de prácticas que nunca ha recibido estudiantes seguramente la empresa se apunte al programa de prácticas en ediciones venideras. 

Otro punto que me gustaría que se mejorara en próximas ediciones es la falta de información por parte de la coordinación de prácticas. No se sabía si sé podían convalidar las prácticas por experiencia laboral, no sabemos cual es la fecha de entrega de esta memoria, he tenido que conocer por los documentos oficiales de aceptación quien es mi tutor académico, con el que nunca antes he hablado y que además afirma no ser el tutor real.

El realizar una prácticas me parece un punto muy importante, casi obligatorio, en la formación de un estudiante de informática. Ya que acercará el mundo de la empresa al de la universidad, esto es algo que muchas veces no se ve claro como estudiantes, todos hemos pensado alguna vez para que valía esa cosa que estabas estudiando y si realmente alguien la utilizaba. Estas prácticas pueden aclarar bastantes conceptos y ayudan a percatarse de las habilidades de uno mismo en un entorno real.

Por último me gustaría agradecer la plantilla en \LaTeX utilizada para el desarrollo de esta memoria \cite{plantilla_latex}.

