\chapter{Valoración personal del trabajo realizado}

El desarrollo de estas prácticas me ha supuesto una experiencia muy interesante y mi primer acercamiento real al mundo empresarial, ya que nunca había trabajado o hecho prácticas como ingeniero informático.

Me ha servido para ver como se trabaja fuera de la universidad, como por ejemplo comprobar desde dentro y con un punto de vista práctico algunas metodologías de desarrollo ágiles que solo había estudiado teóricamente, darme cuenta de como algunas tecnologías aprendidas a lo largo del máster (\textit{Docker}, \textit{ReactJS}, \textit{Flask}, ...) se utilizan realmente o se están implantando en el mundo empresarial, lo que convierte tú formación en un valor adicional, que las empresas valoran.

Centrándome en mi trabajo desarrollado, como es natural al principio estaba un poco perdido, tuve múltiples problemas hasta que comprendí el funcionamiento de las librerías internas de la empresa por la falta de documentación. Una vez empecé a comprender esto mi desarrollo se hizo mucho más fluido, prácticamente al nivel de un miembro del equipo más (llegando incluso a quedarme la última semana solo en el equipo de desarrollo). Por ello puedo decir que estoy contento con mi capacidad de adaptación, ya que en mi escaso período de prácticas he ayudado a desarrollar un proyecto funcional que es actualmente utilizado por la empresa.

Esto mismo ha sido apuntado por algunos de mis responsables directos en la empresa, lo que ha hecho que me hayan tenido en consideración para entrar a formar parte de un nuevo proyecto que comenzará en octubre y me han ofrecido un puesto de trabajo.